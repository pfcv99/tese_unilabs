\chapter{Final remarks}
\label{chapter:Final remarks}

\begin{introduction}
    "The only source of knowledge is experience." - Albert Einstein
\end{introduction}


This work has successfully resulted in the development of a tool designed to capture essential metrics for the quality assessment of \ac{ngs} data. The tool was built with a focus on providing researchers and laboratory technicians with a fast and efficient way to evaluate the quality of \ac{ngs} data. Despite certain limitations, particularly in terms of performance on conventional computing systems, this tool has demonstrated significant potential for further development. By evolving the tool to be compatible with high-performance computing environments, it could be optimized to handle the large volumes of data typically associated with \ac{ngs} analysis, thereby extending its utility and scope.

Throughout this project, several challenges were faced, most notably the performance constraints of the software on standard computing hardware. However, these limitations also offer a clear path for future optimizations and enhancements. The prospect of integrating the software into high-performance computing platforms suggests that with further development, the tool could be adapted to meet the demands of large-scale genomic analysis more effectively. Such advancements would enable the tool to become a valuable asset not only in research contexts but also in clinical and industrial applications where the rapid and accurate evaluation of sequencing data is critical.

This internship has significantly contributed to my growth as a bioinformatician, equipping me with a deeper understanding of bioinformatics and genomics. Engaging with experts in these fields provided me with invaluable knowledge and practical experience, which will undoubtedly be beneficial in my professional future. Moreover, this experience has allowed me to bridge the gap between academic knowledge and its real-world application. The hands-on nature of this project facilitated the application of concepts learned throughout my studies while simultaneously fostering the development of key skills essential for my future career.

In conclusion, this internship has been both enriching and challenging, offering a unique opportunity to contribute to the development of a practical tool that addresses critical needs in genomic data analysis. The experience has not only solidified my expertise in bioinformatics but also opened new avenues for future exploration in the realm of high-performance genomic analysis. As the tool continues to evolve and expand, it holds the promise of becoming an indispensable resource for researchers and clinicians alike, paving the way for faster, more accurate evaluations of \ac{ngs} data.
